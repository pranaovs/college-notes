\documentclass[english,course]{lecture}

% Packages
\usepackage{geometry}
\usepackage{amssymb}
\usepackage{amsmath}
\usepackage{amsthm}

% Page Setting
\geometry{a4paper}

% Custom environments
\newenvironment{qanda}{\begin{enumerate}\setlength{\parindent}{0pt}}{\medskip\end{enumerate}}
\newcommand{\Q}{\bigskip\bfseries \item}
\newcommand{\A}{\par\textbf{A:} \normalfont}

% Headers
\title{Probability and Statistics}
\subtitle{MA1002}
% \shorttitle{Title}
% \subject{Subject of the Talk}
\author{Author}
% \email{Author's email}
\speaker{Speaker}
% \spemail{Speaker's email}
\ccode{Subject Code}
\date{DD}{MM}{YYYY}
% \dateend{}{}{}
% \flag{An extra line if you need it}
% \attn{Something to get reader's attention}
\morelink{}

\begin{document}

\newpage






\section{Basic Concepts}

\subsection{Random Experiment}

We know all the possible outcomes, but do not know the result of the experiment until it is performed.

\textbf{Example:} Tossing a coin, rolling a die, drawing a card from a deck of cards.

\subsection{Outcome}

Result of the random experiment in a single trial.

\subsection{Event}

It is a set of outcomes

\subsection{Equally Likely Event}

There are no reasons o give preference to one outcome over the other.

\textbf{Example:} Tossing a fair coin, rolling a fair die.

\subsection{Mutually Exclusive Events}

Simultaneous occurence of the events is not possible.

\textbf{Example:}
\begin{enumerate}
  \item  Tossing a coin and getting a head and a tail. 
  \item Getting odd and even number on a die.
\end{enumerate}

$A \cup B = \phi$

\subsection{Independent Events}

Occurence of one event does not affect the occurence of the other event.

\textbf{Example:} Tossing a coin two times

$A \cup B = \{HT\} \rightarrow$ Not mutually exclusive

But $A$ and $B$ are Independent

$P(A \cap B) = P(A) \cdot P(B)$


\subsection{Classical Definition of Probability}

In a random experiment, there are $n$ equally likely outcomes.
Let $E$ be the event of interest and $m$ be the number of favourable outcomes in $E$.
Then the probability of event $E$ is given by $P(E) = \frac{m}{n}$

\textbf{Limitations:} 
\begin{itemize}
  \item It is applicable only when all the outcomes are equally likely. 
    \item $n$ may be infinite.
\end{itemize}

\textbf{Example: }
From 25 ticketts, marked with 25 numerals, one ticket is drawn at random. Find the probability that the number on the ticket is a multiple of 3 or 7.


\begin{gather*}
  E = \{3, 6, 9, 12, 15, 18, 21, 24, 7, 14\}\\
  P(E) = \frac{10}{25} = 0.4
\end{gather*}

\subsection{Emperical Reference of Probability}

Consider Random Experiment, which can be repeated in homogeneous condition in $N$ times of trials.
We have got $M$ favourable outcomes for an event $E$.

$\therefore P(E) = {lim}_{N \to \infty}\frac{M}{N} $

\\

\textbf{Limitations: }
\begin{itemize}
  \item Sequence may not converge 
  \item Homogeneous condition may change
\end{itemize}


\subsection{Sample Space}

A set of all possible outcomes of a random experiment is called sample space.

\textbf{Notation:} $S$

Each outcomes if called sample point or elementary event.

\subsection{Event}


An event is a subset of sample space $S$.
$P: \underset{\textbf{Events $P(A)$}}{\text{domain}} \rightarrow \underset{\textbf{$[0,1]$}}{\text{range}}$

\end{document}
