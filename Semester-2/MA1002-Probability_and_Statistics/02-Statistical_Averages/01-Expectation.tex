\documentclass[english,course]{lecture}

% Packages
\usepackage{geometry}
\usepackage{amssymb}
\usepackage{amsmath}
\usepackage{amsthm}

% Page Setting
\geometry{a4paper}

% Custom environments
\newenvironment{qanda}{\begin{enumerate}\setlength{\parindent}{0pt}}{\medskip\end{enumerate}}
\newcommand{\Q}{\bigskip\bfseries \item}
\newcommand{\A}{\par\textbf{A:} \normalfont}

% Headers
\title{Statistical Averages}
\subtitle{Unit 2}
\shorttitle{Expectation}
% \subject{Subject of the Talk}
\author{Pranaov S}
% \email{Author's email}
\speaker{Dr. Jenifer}
% \spemail{Speaker's email}
\ccode{MA1002}
\date{25}{02}{2025}
% \dateend{}{}{}
% \flag{An extra line if you need it}
% \attn{Something to get reader's attention}
\morelink{https://github.com/pranaovs/college-notes}

\begin{document}

\newpage

\section{Expectation}

Expectation talks about the long term.

\subsection{Mathematical Expectation}

If $X$ is a random variable variablw which can assume any one of the values $x_{1}, x_{2}, x_{3}, \dots, x_{n}$
with respective probabilities $p_1, p_2, p_3, \dots, p_n$, then the expectation of $X$ is defined as:

\[
  E(X) = x_{1}p_1 + x_{2} p_2 + \cdots + x_n p_n = \sum_{}^{}x_i p_i
\]

\\

If $X$ is a continuous random variable with probability density function $f(x)$, then the expectation of $X$ is defined as:

\[
  E(X) = \int_{-\infty}^{\infty} x f(x) dx
\]

\subsection{Properties}

\begin{enumerate}
    \item $E(c) = c$ where $c$ is a constant
    \item $E(cX) = cE(X)$ where $c$ is a constant
    \item $E(aX + b) = aE(X) + b$ where $a$ and $b$ are constants
    \item $E(X + Y) = E(X) + E(Y)$
    \item $E(XY) = E(X) \cdot E(Y)$, if $X$ and $Y$ are \textbf{independent} random variables
\end{enumerate}

\end{document}
