\documentclass[english,course]{lecture}

% Packages
\usepackage{geometry}
\usepackage{amssymb}
\usepackage{amsmath}
\usepackage{amsthm}

% Page Setting
\geometry{a4paper}

% Custom environments
\newenvironment{qanda}{\begin{enumerate}\setlength{\parindent}{0pt}}{\medskip\end{enumerate}}
\newcommand{\Q}{\bigskip\bfseries \item}
\newcommand{\A}{\par\textbf{A:} \normalfont}

% Headers
\title{Probability Distributions}
\subtitle{Unit 3}
\shorttitle{Probability Distributions}
% \subject{Subject of the Talk}
\author{Pranaov S}
% \email{Author's email}
\speaker{Dr. Sahaya Jenifer}
% \spemail{Speaker's email}
\ccode{MA1002}
\date{21}{03}{2025}
% \dateend{}{}{}
% \flag{An extra line if you need it}
% \attn{Something to get reader's attention}
\morelink{https://github.com/pranaovs/college-notes}

\begin{document}

\newpage

\section{Bernoulli Trial}

A Bernoulli trial is a random experiment with exactly two possible outcomes, ``success'' and ``failure'', in which the probability of success is the same every time the experiment is conducted. The outcomes are mutually exclusive and exhaustive.

Examples:
\begin{itemize}
  \item "Heads" or "Tails" in a coin toss.
  \item "Pass" or "Fail" in an exam.
  \item "Defective" or "Non-defective" in a manufacturing process.
  \item "Even" or "Odd" in a dice roll.
\end{itemize}

Suppose that a Bernoulli trial is repeated $n$ times,
independently and each time with the same probability of success, $p$.

The probability of getting exactly $x$ successes in $n$ trials is given by the formula

$$^nC_x \cdot p^x \cdot (1-p)^{n-x}$$
where $^nC_x = \frac{n!}{x! \cdot (n-x)!}$ is the number of ways of choosing $x$ successes from $n$ trials,
and $p$ is the probability of success in each trial.


\section{Binomial Distribution}

A random variable $X$ is said to have a binomial distribution with parameters $n$ and $p$ if its probability mass function is given by:

$$ P(X = x) = ^nC_x \cdot p^x \cdot (1-p)^{n-x} \quad \text{for } x = 0, 1, 2, \ldots, n, \text{ 0 otherwise}$$

Where $n$ is the number of trials and $p$ is the probability of success in each trial, $X$ is the number of successes in $n$ trials.


\end{document}
