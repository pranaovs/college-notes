\documentclass[english,course,fleqn]{lecture}

% Packages
\usepackage{geometry}
\usepackage{amssymb}
\usepackage{amsmath}
\usepackage{amsthm}

% Page Setting
\geometry{a4paper}

% Custom environments
\newenvironment{qanda}{\begin{enumerate}\setlength{\parindent}{0pt}}{\medskip\end{enumerate}}
\newcommand{\Q}{\bigskip\bfseries \item}
\newcommand{\A}{\par\textbf{A:} \normalfont}

% Headers
\title{Eigen Values and Eigen Vectors}
\subtitle{Unit 3}
\shorttitle{Eigen Values and Eigen Vectors}
% \subject{Subject of the Talk}
\author{Pranaov S}
% \email{Author's email}
\speaker{Dr.\ Venugopal}
% \spemail{Speaker's email}
\ccode{MA1001}
\date{19}{10}{2024}
% \dateend{}{}{}
% \flag{An extra line if you need it}
% \attn{Something to get reader's attention}
\morelink{https://github.com/pranaovs/college-notes}

\begin{document}

\newpage

\section{Eigen Values and Vectors}

\subsection{Introduction}

\begin{definition}[Eigen Vector]
  An Eigen vector of a square matrix $A$ is a non-zero vector such that when multiplied by by $A$,
  yields into a scalar multiple $(\lambda)$ of itself.

  Mathematically, it is written as:
  \[
    A \times v = \lambda v
  \]

  The scalar multiple $\lambda$ is known as eigen value associated with eigen vector $v$.
\end{definition}

\begin{definition}[Eigen Values]
  Let $T$ be a linear operator on a vector space $V$.
  A non-zero vector $V \in V$ is called an eigen vector of $T$ is there exists a scalar $\lambda$ such that $T \times v = \lambda v$.

  This scalar $\lambda$ is called eigen value corresponding to eigen vector $v$.
\end{definition}

\subsection{Matrix representation of Linear Transformation}

If a Linear Transformation is represented in matrix oform, then it can easily be analysed using algorithms/computers.
Eigen values and eigen vectors help to study the characteristics of the Linear Transformation.

\\

Let $v$ be a finite dimensional vector space.
A non-empty subset $v$ of $B$ is said to be a basis if $B$ is linear transformation.
$B$ generates $v$.

\\

\textbf{Ordered basis:}

Let $V$ be a finite dimensional vector space.
An ordered basis of $V$ is a basis of $V$ in which the elements are arranged in a specific order.

Example: $\{1,x,x^{2}, x^{3}, \ldots, x^{n}\}$ is the ordered basis of $P_{n}(F)$.

\\

\textbf{Note:} For $R^{3}, B_{1} = \{e_{1},e_{2}, e_{3}\}, B_{2} = \{e_{2},e_{3}, e_{1}\}, B_{3} = \{e_{3},e_{1}, e_{2}\}$, and $B_{1} \ne B_{2} \ne B_{3}$

Let $V$ and $W$ be finite dimensional vector space with ordered basis, $\beta = \{v_{1}, v_{2}, v_{3}, \ldots, v_{n}\}$
and $r = \{w_{1}, w_{2}, w_{3}, \ldots, w_{n}\}$ respectively.

Let $T$ form $V \rightarrow W$ be a linear transformation. Then the vectors $T(v_{1}), T(v_{2}), T(v_{3), \ldots, T(v_{n})}$ can be uniquely expressed as linear combination of
$w_{1}, w_{2}, w_{3}, \ldots, w_{n}$.

\begin{gather*}
  T(v_{1}) = a_{11}w_{1} + a_{21}w_{2} + a_{31}w_{3} + \cdots + a_{m1}w_{m}\\
  T(v_{2}) = a_{12}w_{1} + a_{22}w_{2} + a_{32}w_{3} + \cdots + a_{m2}w_{m}\\
  T(v_{3}) = a_{13}w_{1} + a_{23}w_{2} + a_{33}w_{3} + \cdots + a_{m3}w_{m}\\
  \vdots\\
  T(v_{n}) = a_{1n}w_{1} + a_{2n}w_{2} + a_{3n}w_{3} + \cdots + a_{mn}w_{m}\\
  \implies T(v_{i}) = a_{ji}w_{j}
\end{gather*}

\\

A matrix representation of $T$ with respect to $\beta$ and $\gamma$ is:
\[
  A = {[T]}_{\beta}^{\gamma} = \begin{bmatrix}
    a_{11} & a_{12} & a_{13} & \cdots & a_{1n} \\
    a_{21} & a_{22} & a_{23} & \cdots & a_{2n} \\
    a_{31} & a_{32} & a_{33} & \cdots & a_{3n} \\
    \vdots & \vdots & \vdots & \ddots & \vdots \\
    a_{m1} & a_{m2} & a_{m3} & \cdots & a_{mn} \\
  \end{bmatrix}
\]

where:
\[
  \begin{bmatrix}
    a_{11}  \\ 
    a_{21}  \\ 
    a_{31}  \\ 
    \vdots  \\ 
    a_{m1}  \\ 
  \end{bmatrix} = \text{Coefficient of } T(v_{1})
  \begin{bmatrix}
    a_{12}  \\ 
    a_{22}  \\ 
    a_{32}  \\ 
    \vdots  \\ 
    a_{m2}  \\ 
  \end{bmatrix} = \text{Coefficient of } T(v_{2})
  \ldots
  \begin{bmatrix}
    a_{1n}  \\ 
    a_{2n}  \\ 
    a_{3n}  \\ 
    \vdots  \\ 
    a_{mn}  \\ 
  \end{bmatrix} = \text{Coefficient of } T(v_{n})
\]

\textbf{Note:}
\begin{enumerate}
  \item Every linear transformation $T$ can be associated with a $m \times n$ matrix, and the converse is also true.
  \item $T:v \rightarrow V$, then the matrix of $T$ with respect to basis $\beta$ is $A = {[T]}_{\beta}$.
  \item The image of vector $v$ in $V$ is $T(v) = {[T]}_{\beta}^{n}v^{T}$.
\end{enumerate}

\subsubsection{Examples}

\begin{qanda}

  \Q Obtain a matrix representing a linear transformation $T:V_{3}(R) \rightarrow V_{3}(R)$
  given by $T(a,b,c) = (3a, a-b, 2a + b + c)$ with respect to standard basis

  \A 

  \begin{gather*}
    \text{Basis of } V_{3}(R) = \beta = \{e_{1}, e_{2}, e_{3}\} = {(1,0,0), (0,1,0), (0,0,1)}\\
    T(e_{1}) = (3,1,2) = 3e_{1} + e_{2} + e_{3}\\
    T(e_{2}) = (0,-1,1) = 0e_{1} - e_{2} + e_{3}\\
    T(e_{3}) = (0,0,1) = 0e_{1} + 0e_{2} + e_{3}\\
    \\
    \text{Matrix representation of $T$ is:}\\
    A = {[T]}_{\beta} = \begin{bmatrix}
      3 & 0 & 0 \\
      1 & -1 & 0 \\
      2 & 1 & 1 \\
    \end{bmatrix}
  \end{gather*}

  \Q Let $T:V_{3}(R) \rightarrow V_{2}(R)$ given by $T(a,b,c) = (a+b, 2c-a)$. Find the matrix representation with respect to standard basis.

  \A 

  \begin{gather*}
    \text{Basis of } V_{3}(R) = \beta = \{e_{1}, e_{2}, e_{3}\} = {(1,0,0), (0,1,0), (0,0,1)}\\
    T(e_{1}) = (1,-1) = e_{1}' - e_{2}'\\
    T(e_{2}) = (1,0) = e_{1}' + e_{2}'\\
    T(e_{3}) = (0,2) = 0e_{1}' + 2e_{2}'\\
    \\
    A = {[T]}_{\beta}^{\gamma} = \begin{bmatrix}
      1 & 1 & 0 \\
      -1 & 0 & 2 \\
    \end{bmatrix}
  \end{gather*}

  \Q Find the matrix of the Linear Transformation $T:P_{3}(R) \rightarrow P_{2} (R)$ defined by $T(F(x)) = F'(X)$ with respect to standard basis

  \A \begin{gather*}
    B = \{e_{1}, e_{2}, e_{3}\}\\
    e_{1} = {1}\\
    e_{2} = {x}\\
    e_{3} = {x^{2}}\\
    e_{4} = x^{3}\\
    \\
    r = \{e_{1}', e_{2}', e_{3}'\}\\
    \\
    T(e_{1}) = 0 = 0e_{1}' + 0e_{2}' + 0e_{3}'\\
    T(e_{2}) = 1 = 1e_{1}' + 0e_{2}' + 0e_{3}'\\
    T(e_{3}) = 2x = 0e_{1}' + 2e_{2}' + 0e_{3}'\\
    T(e_{4}) = 3x^{2} = 0e_{1}' + 0e_{2}' + 3e_{3}'\\
    \\
    A = {[T]}_{\beta} = \begin{bmatrix}
      0 & 1 & 0 & 0 \\
      0 & 0 & 2 & 0\\
      0 & 0 & 0 & 3 \\
    \end{bmatrix}
    \end{gather*}
  \end{qanda}

  \newpage

  \lecture[]{23}{10}{2024}

  \begin{theorem}[]
    Let $T$ be the lienar operator on a finite dimensional vector space $B$ and $\beta$ is an ordered basis for $B$.
    Then, $\lambda$ is called an eign value of $T \iff \lambda$ is called eigen value of its matrix $T (A = {[T]}_{\beta})$.
  \end{theorem}

  \subsection{Method to find eigen values andeigen vectors}

  \begin{enumerate}
    \item \textbf{Form characteristic equation for the given matrix}

      i.e. $|A-\lambda I| = 0$ \margintext{\begin{gather*}
          A \times v = \lambda v\\
          \implies A\times v - \lambda v = 0\\
          \implies (A - \lambda I) v = 0\\
          \implies |A - \lambda I| = 0
      \end{gather*}}


    \item \textbf{Solve the characteristic equation}

      The solution of the equation are known as \textbf{eigen values of the matrix}.

      \begin{gather*}
        A = \begin{bmatrix}
          a_{11} & a_{12} \\
          a_{21} & a_{22} \\
        \end{bmatrix}\\
        A - \lambda I =  \begin{bmatrix}
          a_{11} & a_{12} \\
          a_{21} & a_{22} \\
          \end{bmatrix} - \lambda \begin{bmatrix}
          1 & 0 \\
          0 & 1 \\
        \end{bmatrix}\\
        \implies \begin{bmatrix}
          a_{11} - \lambda & a_{12} \\
          a_{21} & a_{22} - \lambda \\
        \end{bmatrix}
      \end{gather*}

      Characteristic equation is

      \begin{gather*}
        |A - \lambda I| = 0\\
        \begin{vmatrix}
          a_{11} - \lambda & a_{12} \\
          a_{21} & a_{22} - \lambda \\
        \end{vmatrix} = 0\\
        \lambda^{2} - (a_{11} + a_{12})\lambda + (a_{11}a_{12} - a_{21} a _{22}) = 0
      \end{gather*}

    \item \textbf{Corresponding to each eigen value $\lambda_{i}$ an eigen vector $v_{i}$ is obtained by solving the matrix equation $(A - \lambda_{i}I) v_{i} = 0$}
  \end{enumerate}

  \subsection{Method to form characteristic equation}


  Let $A$ be a square matrix of order $n$.
  \[
    A = \begin{bmatrix}
      a_{11} & a_{12} & a_{13} & \cdots & a_{1n} \\
      a_{21} & a_{22} & a_{23} & \cdots & a_{2n} \\
      a_{31} & a_{32} & a_{33} & \cdots & a_{3n} \\
      \vdots & \vdots & \vdots & \ddots & \vdots \\
      a_{m1} & a_{m2} & a_{m3} & \cdots & a_{mn} \\
    \end{bmatrix}
  \]

  Characteristic equation is $A - \lambda I = 0$
  \[
    A = \begin{vmatrix}
      a_{11} - \lambda & a_{12} & a_{13} & \cdots & a_{1n} \\
      a_{21} & a_{22} - \lambda & a_{23} & \cdots & a_{2n} \\
      a_{31} & a_{32} & a_{33} - \lambda & \cdots & a_{3n} \\
      \vdots & \vdots & \vdots & \ddots & \vdots \\
      a_{m1} & a_{m2} & a_{m3} & \cdots & a_{mn} - \lambda \\
    \end{vmatrix} = 0
  \]

  Expanding: \[
    (-1)^{n} \lambda^{n} + K_{1}\lambda^{n-1} + K_{2}\lambda^{n-2} + \cdots + K_{n} = 0
  \]
  where $K_{i}$'s are expressible in terms of elements of $a_{ij}$.

  \subsubsection*{Shortcuts}

  \begin{enumerate}
    \item Let $A$ be a $2 \times 2$ matrix.
      \[
        A = \begin{bmatrix}
          a_{11} & a_{12} \\
          a_{21} & a_{22} \\
        \end{bmatrix}
      \]
      \[
        | A - \lambda I| = \lambda^{2} - S_{1}\lambda + S_{2} = 0
      \]
      \begin{gather*}
        S_{1} =  \text{Trace of matrix } A = a_{11} + a_{22}\\
        S_{2} = |A|
      \end{gather*}

    \item Let $A$ be a $3 \times 3$ matrix
      \[
        A = \begin{bmatrix}
          a_{11} & a_{12} & a_{13} \\
          a_{21} & a_{22} & a_{23} \\
          a_{31} & a_{32} & a_{33} \\
        \end{bmatrix}
      \]
      \[
        |A - \lambda I| = \lambda^{3} - S_{1}\lambda^{2} + S_{2} \lambda - S_{3}
      \]

      \begin{gather*}
        S_{1} = tr(A)\\
        S_{2} = \text{Sum of minors of leading diagonal elements}\\
        = \begin{vmatrix}
          a_{22} & a_{23} \\
          a_{32} & a_{33} \\
          \end{vmatrix} + \begin{vmatrix}
          a_{11} & a_{13} \\
          a_{31} & a_{33} \\
          \end{vmatrix} + \begin{vmatrix}
          a_{11} & a_{12} \\
          a_{21} & a_{22} \\
        \end{vmatrix}\\
        S_{3} = |A|
      \end{gather*}
  \end{enumerate}

  \subsubsection{Questions}

  \begin{qanda}
    \Q Find eigen value and eigen vectors of $A = \begin{bsmallmatrix}
      5 & 4 \\
      1 & 2 \\
    \end{bsmallmatrix}$

    \A The chatacteristic equation of $A$ is:
    \begin{gather*}
      |A = \lambda I| = 0\\
      \implies \lambda^{2} + S_{1} \lambda + S_{2} = 0\\
      S_{1} = tr(A) = 7\\
      S_{2} = |A| = 6
      \\
      \therefore |A - \lambda I| = \lambda^{2} - 7 \lambda + 6 = 0\\
      \lambda_{1} = 6;\lambda_{2} = 1
    \end{gather*}

    To find eigen vectors:
    \begin{gather*}
      (A - \lambda_{i}I)v_{i} = 0\\
      \begin{bmatrix}
        5 - \lambda & 4 \\
        4 & 2 - \lambda \\
        \end{bmatrix} \begin{bmatrix}
        x_{1} \\
        x_{2} \\
        \end{bmatrix} = \begin{bmatrix}
        0 \\
        0 \\
      \end{bmatrix}\\
      \text{Case 1: } \lambda_{1} = 6\\
      \begin{bmatrix}
        5 - 6 & 4 \\
        4 & 2 - 6 \\
        \end{bmatrix} \begin{bmatrix}
        x_{1} \\
        x_{2} \\
        \end{bmatrix} = \begin{bmatrix}
        0 \\
        0 \\
      \end{bmatrix}\\
      = \begin{bmatrix}
        -1 & 4 \\
        4 & -4 \\
        \end{bmatrix} \begin{bmatrix}
        x_{1} \\
        x_{2} \\
        \end{bmatrix} = \begin{bmatrix}
        0 \\
        0 \\
      \end{bmatrix}\\
      \\
      -x_{1} + 4 x_{2} = 0\\
      x_{1} - 4 x_{2} = 0\\
      \implies x_{1} = 4 x_{2}\\
      \text{Let } x_{2} = 1; x_{1} = 4
    \end{gather*}

    Eigen vector is $v_{1} = \begin{bsmallmatrix}
      4 \\
      1 \\
    \end{bsmallmatrix}$

    $||| ~ v_{2} = \begin{bmatrix}
      -1 \\
      1 \\
    \end{bmatrix}$
  \end{qanda}

  \end{document}
